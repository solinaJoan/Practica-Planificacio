\section{Operadors}

El domini defineix dues accions principals (\texttt{assignar-reserva} i \texttt{descartar-reserva}) que permeten al planificador construir solucions vàlides al problema d'assignació de reserves. 

\vspace{0.3cm}

No es requereixen operadors addicionals perquè el problema no presenta accions més complexes: totes les decisions es redueixen a assignar o no assignar cada reserva, i el planificador pot explorar diferents combinacions d'aquestes decisions per trobar la solució òptima segons el criteri especificat.

\vspace{0.3cm}

\subsection{Acció: assignar-reserva}

Aquesta és l'acció principal del domini i està present en totes les extensions. Permet assignar una reserva pendent a una habitació que compleix els requisits necessaris.

\vspace{0.3cm}

\textbf{Paràmetres:}
\begin{itemize}
    \item \texttt{?r - reserva}: la reserva a assignar.
    \item \texttt{?h - habitacio}: l'habitació on s'assignarà la reserva.
\end{itemize}

\vspace{0.3cm}

\textbf{Precondicions:}

\begin{enumerate}
    \item \texttt{(pendent ?r)}: la reserva ha d'estar pendent de processar. Aquesta precondició és fonamental per evitar assignar la mateixa reserva múltiples vegades.
    
    \item \texttt{(>= (capacitat ?h) (num-persones ?r))}: l'habitació ha de tenir capacitat suficient per allotjar totes les persones de la reserva. Aquesta comprovació és essencial per garantir que la solució és físicament viable.
    
    \item \texttt{(forall (?d - dia) (imply (dia-reserva ?r ?d) (not (ocupada ?h ?d))))}: cap dels dies de la reserva pot coincidir amb una ocupació prèvia de l'habitació. Aquest quantificador universal és crucial per evitar solapaments temporals entre reserves a la mateixa habitació.
\end{enumerate}

Aquestes precondicions són necessàries i suficients per garantir que qualsevol assignació realitzada compleix amb les restriccions dures del problema (capacitat i no solapament).

\vspace{0.3cm}

\textbf{Efectes:}

\begin{enumerate}
    \item \texttt{(assignada ?r ?h)}: estableix la relació d'assignació entre la reserva i l'habitació. Aquest efecte és necessari per mantenir el registre de les assignacions.

    \item \texttt{(not (pendent ?r))}: elimina la reserva de les pendents. Aquest efecte serveix per evitar reprocessar la mateixa reserva. i sobretot per crear l'objectiu (goal) del problema, que requereix que totes les reserves hagin estat processades almenys un cop. 

    \item \texttt{(forall (?d - dia) (when (dia-reserva ?r ?d) (ocupada ?h ?d)))}: marca tots els dies de la reserva com a ocupats a l'habitació assignada. Aquest efecte condicional és imprescindible per prevenir assignacions futures que causin solapaments.
\end{enumerate}

\vspace{0.3cm}

\textbf{Efectes específics per extensions:}

A l'extensió 2, s'afegeix un efecte addicional per comptabilitzar les orientacions incorrectes:

\begin{itemize}
    \item \texttt{(forall (?o - orientacio) (when (and (te-orientacio ?h ?o) (not (prefereix-orientacio ?r ?o))) (increase (orientacions-incorrectes) 1)))}: si l'orientació de l'habitació no coincideix amb la preferència de la reserva, s'incrementa el comptador d'orientacions incorrectes.
\end{itemize}

A l' extensió 3, s'afegeix l'efecte per comptabilitzar el desperdici de places:

\begin{itemize}
    \item \texttt{(increase (places-lliures) (- (capacitat ?h) (num-persones ?r)))}: incrementa el comptador de places lliures amb la diferència entre la capacitat de l'habitació i el nombre de persones de la reserva.
\end{itemize}

A l'extensió 4, s'afegeix l'efecte per controlar l'obertura d'habitacions:

\begin{itemize}
    \item \texttt{(when (not (habitacio-oberta ?h)) (and (habitacio-oberta ?h) (increase (habitacions-obertes) 1)))}: si és la primera vegada que s'utilitza l'habitació, es marca com a oberta i s'incrementa el comptador d'habitacions obertes.
\end{itemize}

\vspace{0.3cm}

\subsection{Acció: descartar-reserva}

Aquesta acció s'introdueix a l'extensió 1 i es manté a totes les extensions posteriors. Permet al planificador deixar reserves sense assignar quan no és possible trobar una assignació vàlida o quan assignar-les empitjoraria excessivament la funció objectiu. Al nivell bàsic, simplement no soluciona el problema i no dona cap assignació

\vspace{0.3cm}

\textbf{Paràmetres:}
\begin{itemize}
    \item \texttt{?r - reserva}: la reserva a descartar.
\end{itemize}

\vspace{0.3cm}

\textbf{Precondicions:}

\begin{itemize}
    \item \texttt{(pendent ?r)}: la reserva ha d'estar pendent. Aquesta és l'única precondició necessària, ja que qualsevol reserva pendent pot ser descartada.
\end{itemize}

\vspace{0.3cm}

\textbf{Efectes:}

\begin{enumerate}
    \item \texttt{(not (pendent ?r))}: elimina la reserva de les pendents, permetent complir l'objectiu de l'extensió 1 que requereix que no hi hagi reserves pendents.
    
    \item \texttt{(increase (reserves-assignades) 1)}: incrementa el comptador de reserves no assignades. 
\end{enumerate}


