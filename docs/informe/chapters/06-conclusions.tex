\section{Conclusions}

\textbf{Sobre l'experimentació} L'anàlisi experimental revela que la complexitat del problema d'assignació de reserves creix de manera exponencial, especialment quan s'incrementa el nombre de reserves. El creixement simultani d'ambdues dimensions d'habitacions i reserves produeix l'escalada més pronunciada en el temps de planificació. Els resultats suggereixen la necessitat d'utilitzar heurístiques o tècniques de descomposició per abordar problemes de major escala. Fer servir un generador aleatori ha produït que tinguem problemes semblants a la realitat difícils de reproduir. També ha fet que alguns problemes siguin exageradament difícils i d'altres resolts més ràpidament tot i ser un poblema de més tamany