\section{Problema}

\subsection{Estat Inicial}

L'estat inicial defineix tota la informació coneguda al començament del problema. Bàsicament hem de declarar els objectes del problema (dies, habitacions i reserves) i incialitzar paràmetres

\begin{enumerate}
    \item \textbf{Capacitat de les habitacions}: mitjançant la funció numèrica \texttt{(= (capacitat ?h) N)}, s'especifica el nombre màxim de persones que pot allotjar cada habitació.
    
    \item \textbf{Característiques de les reserves}: per cada reserva es defineix:
    \begin{itemize}
        \item \texttt{(= (num-persones ?r) N)}: nombre de persones de la reserva.
        \item \texttt{(pendent ?r)}: predicat que indica que la reserva està pendent de processar.
        \item \texttt{(dia-reserva ?r ?d)}: conjunt de predicats que estableixen quins dies ocupa la reserva.
    \end{itemize}
    \item \textbf{ Inicialitzar a zero els contadors} necessàris per optimitzar la solució
\end{enumerate}

\vspace{0.3cm}

\textbf{Elements afegits per extensions:}


A l'extensió 2, s'afegeix informació sobre orientacions, tant de les preferides per la reserva com la de les habitacions:
\begin{itemize}
    \item \texttt{(te-orientacio ?h ?o)}: indica l'orientació de cada habitació.
    \item \texttt{(prefereix-orientacio ?r ?o)}: indica la preferència d'orientació de cada reserva.
\end{itemize}

\vspace{0.3cm}

\subsection{Estat Final (Goal)}

L'objectiu del problema és idèntic en totes les extensions i requereix que totes les reserves hagin estat processades:

\begin{itemize}
    \item \texttt{(forall (?r - reserva) (not (pendent ?r)))}
\end{itemize}

Aquest objectiu obliga el planificador a prendre una decisió per cada reserva, ja sigui assignant-la a una habitació (acció \texttt{assignar-reserva}) o descartant-la (acció \texttt{descartar-reserva}, disponible a partir de l'extensió 1). Al nivell bàsic, sense l'acció de descartar, el planificador ha de trobar assignacions vàlides per a totes les reserves o el problema no tindrà solució.

\vspace{0.3cm}

\subsection{Mètrica}

La mètrica defineix el criteri d'optimització que guiarà el planificador cap a la millor solució possible. Cada extensió incorpora nous factors a considerar:

\vspace{0.3cm}

\textbf{Nivell bàsic:} No hi ha mètrica. El planificador només ha de trobar qualsevol pla vàlid que compleixi l'objectiu.

\vspace{0.3cm}

\textbf{Extensió 1:} S'introdueix la primera mètrica per minimitzar les reserves no assignades:

\begin{itemize}
    \item \texttt{(:metric minimize (reserves-assignades))}
\end{itemize}

El valor del comptador s'incrementa cada reserva descartada, penalitzant les solucions que no assignen reserves.

\vspace{0.3cm}

\textbf{Extensió 2:} La mètrica combina dos objectius amb pesos diferents:

\begin{itemize}
    \item \texttt{(:metric minimize (+ (orientacions-incorrectes) (* 100 (reserves-assignades))))}
\end{itemize}

El pes de 100 sobre les reserves no assignades garanteix que primer es prioritza assignar totes les reserves possibles, i només després s'optimitza la satisfacció de preferències d'orientació. 

\vspace{0.3cm}

\textbf{Extensió 3:} S'afegeix l'objectiu de minimitzar el desperdici de places:

\begin{itemize}
    \item \texttt{(:metric minimize (+ (places-lliures) (* 100 (reserves-assignades))))}
\end{itemize}

El comptador \texttt{places-lliures} s'incrementa amb la diferència entre la capacitat de l'habitació i el nombre de persones de cada reserva assignada. El pes de 100 manté la prioritat d'assignar reserves sobre l'eficiència d'ocupació.

\vspace{0.3cm}

\textbf{Extensió 4:} La mètrica final incorpora tres criteris jeràrquics:

\begin{itemize}
    \item \texttt{(:metric minimize (+ (places-lliures) (+ (* 100 (habitacions-obertes)) (* 1000 (reserves-assignades)))))}
\end{itemize}


Els pesos estableixen una jerarquia de decisió clara: es prioritza assignar reserves, minimitzant el nombre d'habitacions obertes, i finalment l'últim criteri és l'aprofitament de places i ajustar el nombre de persones a la reserva amb la capacitat de l'habitació. Els pesos fan que tot i que haguem d'obrir una habitació nova i penalitzi, sempre surti més rentable que rebutjar una assignació vàlida, o que igualment que tinguem habitacions ajustades per totes les reserves, no les obrim totes. 

Aquesta estructura permet al planificador concentrar les reserves en el menor nombre d'habitacions possible sense sacrificar assignacions.