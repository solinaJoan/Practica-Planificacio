\section{Conclusions}

\textbf{Sobre la pràctica}, ha funcionat molt el mètode incremental, primer implementant el nivell bàsic del problema i poc a poc anar afegint les coses necessàries per les extensions. Ens ha costat sobretot al principi entendre el software de \textbf{metric-ff} i com funciona l'optimització dels problemes, però ens n'hem acabat sortint.

\vspace{0.3cm}

\textbf{Sobre l'experimentació}, l'anàlisi experimental revela que la complexitat del problema d'assignació de reserves creix de manera exponencial, especialment quan s'incrementa el nombre de reserves. El creixement simultani d'ambdues dimensions d'habitacions i reserves produeix l'escalada més pronunciada en el temps de planificació. Els resultats suggereixen la necessitat d'utilitzar heurístiques o tècniques de descomposició per abordar problemes de major escala. Fer servir un generador aleatori ha produït que tinguem problemes semblants a la realitat difícils de reproduir. També ha fet que alguns problemes siguin exageradament difícils i d'altres resolts més ràpidament tot i ser un poblema de més tamany