\section{Jocs de Prova}
Tenim dos jocs de prova per cada problema. Un generat per testejar que la solució sigui correcta, i un generat per el generador de problemes. Tots els problemes

'assignar reserves únicament a habitacions amb capacitat suficient, evitar solapaments temporals entre reserves assignades a la mateixa habitació i 


\subsection{Extensió 1}
\paragraph{Problema 1 Extensió 1:}
El problema proposat consta de \textbf{4 habitacions} i \textbf{6 reserves}. 

Hi ha una reserva gran (r4) que només pot ser assignada a una única habitació i que bloqueja aquest recurs durant molts dies.

\paragraph{Objectiu del joc de prova}

Aquest joc de prova té com a objectiu comprovar que el planificador és capaç de seleccionar un subconjunt òptim de reserves quan no és possible assignar-les totes.

L’optimització es basa exclusivament en \textbf{maximitzar el nombre total de reserves assignades}, permetent que algunes reserves quedin sense assignar si això condueix a una millor solució global.

\paragraph{Resultat esperat}

S’espera que la solució òptima rebutgi la reserva gran i de llarga durada, ja que acceptar-la impediria assignar diverses reserves més petites. En conseqüència, el planificador hauria de prioritzar l’assignació de reserves curtes i de menys persones, assolint un nombre total de reserves assignades superior al que s’obtindria acceptant totes les reserves de gran mida.

\paragraph{Resultat real}
El planificador dona aquest resultat, que és el que esperàvem: 
\begin{lstlisting}[language=Lisp]
0: DESCARTAR-RESERVA R4
1: ASSIGNAR-RESERVA R1 H1
2: ASSIGNAR-RESERVA R2 H2
3: ASSIGNAR-RESERVA R3 H3
4: ASSIGNAR-RESERVA R5 H2
5: ASSIGNAR-RESERVA R6 H1

\end{lstlisting}



\paragraph{Problema 2 Extensió 1: }

Generat amb el generador de problemes, amb 4 habitacions i 10 reserves.
